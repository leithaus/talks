\part{From calculi to language design}
\chapter{Data and control}
\section{Names and messages}

Having derived a closed account of names it is hard to miss the fact
that names have an incredible amount of structure. In fact, remarkably
the structure is not terribly far from that of \texttt{XML} documents
as noted in \cite{}. The real issue is in making this structure
accessible to processes. In point of fact, we already have mechanism
to construct names. The drop operator acts as a kind of splicing
mechanism (ala \texttt{Lisp} or \texttt{Scheme} macros), while the
lift operator actually produces names from process terms (that may
have been partially constructed from splicing). What is missing is a
means of taking them apart. Taking a page from a robust and
increasingly popular language (\texttt{Erlang}) which, in turn, took a
page from the functional languages, the natural place to introduce
this \emph{destructuring} mechanism is in \emph{input}. In turn,
wrapping this \emph{pattern-matching} behavior in a case-like
construct -- which just happens to fit with the notion of summation --
is also a natural design choice. The only real wrinkle in this
approach is the issue of pattern-matching patterns.

\subsection{Grammar}

\begin{figure}[hbp]
  \centering
  \Ovalbox{
    \begin{mathpar}
      \inferrule* [lab=Process,leftskip=2em] {} {P \bc
        \textsf{case}\textsf{\{} u_1 A_1 \textsf{;} \;\ldots \textsf{;}\;
        u_n A_n \textsf{\}} \;\;|\;\; \textsf{\{} P_1 \textsf{;} \;\ldots
        \textsf{;}\; P_n \textsf{\}} \;\;|\;\; \textsf{@}u }
      \and \\
      \inferrule* [lab=Agent] {} {A \bc \textsf{?} \textsf{(} a_1
        \textsf{,} \ldots \textsf{,} a_n \textsf{)}\textsf{.}P \;\;|\;\;
        \textsf{!} \textsf{(} a_1 \textsf{,} \ldots \textsf{,} a_n
        \textsf{)} \textsf{:=} \textsf{(} P_1 \textsf{,} \ldots \textsf{,}
        P_n \textsf{)}\textsf{.}P}
      \and \\
      \inferrule* [lab=Pattern] {} {a \bc e \;\;|\;\; q}
      \and
      \inferrule* [lab=Query] {} {q \bc s \;\;|\;\; \textsf{\{}a_1 \textsf{;} \ldots
        \textsf{;} a_n\textsf{\}} \;\;|\;\; \textsf{@}\langle\langle
        a\rangle\rangle \;\;|\;\; \textsf{rec}\;id \textsf{.}a }
      \and \inferrule* [lab=Sum] {} {s \bc \textsf{\{}\textsf{\}}
        \;\;|\;\; v \;\;|\;\; u \; g + s}
      \and
      \inferrule* [lab=Guard] {} {g \bc \textsf{?} t\textsf{.}q \;\;|\;\;
        \textsf{!} t \textsf{:=} t'\textsf{.}q \;\;|\;\;
        \textsf{*}v\textsf{.}q}
      \and \\
      \inferrule* [lab=Tuple] {} {t \bc \textsf{(}\textsf{)} \;\;|\;\; v
        \;\;|\;\;\textsf{(} m_1 \textsf{,} \ldots \textsf{,} m_n
        \textsf{)} \;\;|\;\; a \textsf{::} t} \and \inferrule*
      [lab=Nesting] {} {m \bc a \;\;|\;\; t}
      \and \\
      \inferrule* [lab=Site] {} {u \bc x \;\;|\;\; v} \and \inferrule*
      [lab=Code] {} {x \bc \langle\langle P \rangle\rangle}
      \and
      \inferrule* [lab=Variable] {} {v \bc id \;\;|\;\; \textsf{\_}}
      \and \\
      \inferrule* [lab=Value] {} {e \bc x \;\;|\;\; bool \;\;|\;\; nat \;\;|\;\; \ldots} 
  % \inferrule* [lab=Expression] {} {e \bc l \;\;|\;\; logical \;\;|\;\; numeric \;\;|\;\; \ldots} 
%   \and \\
%   \inferrule* [lab=Literal] {} {l \bc x \;\;|\;\; bool \;\;|\;\; nat \;\;|\;\; \ldots}
%   \and \\
%   \inferrule* [lab=Identifier] {} {id \bc n c_1 \ldots c_n}
%   \and \\
%   \inferrule* [lab=AlphaNum] {} { n \bc \textsf{'A'}\;|\;\ldots\;|\;\textsf{'z'} \;|\; \textsf{'0'}\;|\;\ldots\;|\;\textsf{'9'} }
%   \and \\
%   \inferrule* [lab=AlphaPlus] {} {c \bc l \;\;|\;\; \textsf{\_} \;\;|\;\; \textsf{-} \;\;|\;\; \ldots}
    \end{mathpar}
  }
  \caption{ holOgram abstract syntax }
\end{figure}