\section{Conclusions and future work}

We have exhibited a framework in which to reason about interaction
across scale. The framework derives from a careful analysis of
Milner's decomposition of the sequencing operator in the
$\pi$-calculus and the roles of its components in the comm rule, the
engine of the calculus and the formal counterpart to Milner's
intuition that computation is interaction.

In the context of deliberating the relationship between scale and
complexity we wish to conclude by calling to attention one of the
cornerstones of physical theory in this connection. Current thinking
presupposes that ``smaller'' really does mean simpler, and yet when
this is pushed to the utmost limit we find proposals very much at odds
with this attitude seriously debated by the scientific
community. Specifically, the attempt to ``reduce'' the complexity of
the standard model of physics and reconcile quantum mechanics with
gravitational theories calls upon entities, such as the vibratory
modes of strings, which are rich enough, from a computational
perspective, to encode universal Turing machines. This begs the
question as to how much ``simplification'' has been
accomplished. Rather than seeing this as an internal inconsistency we
view this as an exciting possibility that smaller \emph{doesn't} mean
simpler. That the two notions are really orthogonal.

It seems likely that an atomistic view of the world as bottoming out
in structure at some scale may have led to the identification of these
concepts. Further, what may have been at the crux of the debate that
the world must eventually ``bottom out'' in some smallest something
was (is) an inability to divorce discrete structure from atomistic
structure. Without the conceptual tools of computational and symbolic
dynamics it seems very hard to imagine discrete, but infinitely
recursive, infinitely scaling structures, where what is discrete are
the constructors, not the extent.

% section conclusions and future work (end)


