\def\lastname{Meredith}

\title{Interaction across scale: what the butterfly wings say to the hurricane}
\titlerunning{Interaction across scale}

\author{ L.G. Meredith\inst{1} }
\institute{ Partner, Biosimilarity\\ 505 N72nd St, Seattle, WA 98103, USA, \\
  \email{ lgreg.meredith@biosimilarity.com } 
} 

\maketitle              % typeset the title of the contribution

%%% ----------------------------------------------------------------------

\begin{abstract}

  % One of the remarkable and most difficult-to-model aspects of
%   physical processes is the manner in which entities or subprocesses
%   occupying spatio-temporal scales separated by of orders of magnitude
%   interact to achieve an effect. Signaling processes in the cell,
%   where subsecond length processes in the cytoplasm interact with
%   processes in the nucleus taking place over minutes, provide a prime
%   example. We present, here, a model of interaction, derived from the
%   mobile process calculi allowing for a general, compositional and
%   quantitative treatment of interaction across scale.

  We present, here, a model of interaction, derived from the mobile
  process calculi, allowing for a general, compositional and
  quantitative treatment of interaction across scale.

  % Today, the treatment of the dynamics of physical systems is markedly
%   different from the treatment of the dynamics of symbolic or
%   computational systems. The dominant apparatus for the formulation of
%   physical theory and the elaboration and analysis of the dynamics of
%   physical systems is Newton's calculus, while it is at least arguable
%   that compositional systems such as the $lambda$-calculus and the
%   $pi$-calculus are beginning to dominate the formulation of
%   computational theories and the elaboration and analysis of the
%   dynamics of such systems. Yet, the challenges facing both kinds of
%   endeavors indicated that a convergence is in the cards. Whether one
%   is analyzing biological processes, like cytoplasmic signaling
%   processes interacting with gene-regulatory processes taking place in
%   the nucleus, for the practical purpose of devising therapeutic
%   methods, or grand endeavors like unifying the physical theories of
%   the quantum interaction with the physical theories of gravitation,
%   mathematical accounts of the dynamics physical systems are stretched
%   to the breaking point to deal with issues if scale. Likewise,
%   accounts of computational dynamics are being forced to deal more and
%   more with the details of their realization as physical processes,
%   from issues of utilization of resources to locality and
%   simultaneity.

\end{abstract}

%\keywords{Process calculi, knots, invariants}

% \begin{keyword}
% concurrency, message-passing, process calculus, reflection, program logic
% \end{keyword}

%\end{frontmatter}