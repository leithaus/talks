Web services are gaining momentum as a platform, both conceptual and
technological, upon which to develop applications that take advantage
of the Internet infrastructure. The popularity of standards and
proposed standards like XML, WSDL [4], WS-ReliableMessaging [2],
WS-Security [2], BEPL4WS [5], WSCI [1], and others lend credence to
this claim. Much to their credit, these and other standards have led
to the development of a basic substrate of tools for wiring together
the nodes of a distributed application. For example, Microsoft's .Net
offering provides support for XML and WSDL, and its WSE 1.0 supports
many WS-* standards and proposals. 

On the other hand, by and large, these standards say little about the
correct functioning of such an application. WSDL, for example,
only addresses the message format and transport bindingsat (aggregates
of) endpoints. This situation is unfortunate because much of the
substantive complexity of distributed heterogenous applications lies in
the interoperation of the services, not the connectivity. All the issues
of safety and liveness, for example,arise only after connectivity
among the nodes hasbeen addressed. Put another way, even in a world
of perfect connectivity with perfect, synchronous communication among
perfectly trusted peers, safety and liveness are still thorny
issues. And, they are made thornier still in a dynamic environment,
suchas the Internet, where services discover and interact with each
other over the course of a computation.  

The situation is exacerbated by proposals like BEPL4WS and WSCI because
these proposals do address behavioral aspects of services. From
the point of view of correctness, however, these proposals say too
much. Both offer Turing-complete languages for orchestrating behavior
across ensembles of services. The expressive power of these
languages causes them to be more about implementation than specification
of useful and interesting properties. At best, these languages, had
they formal semantics, might be used for model-checking-style
verification. Of course, the worry of a model-checking approach is that
one may simply run out of time or space and have no idea whether a
given service operates according to the model. Further, throwing
more `iron' at the problem, may or may not improve
the situation. Finally, exposing implementation even at the levels of
abstraction, such as they are, provided by BEPL4WS or WSCI is arguably
not the role of a description language, nor of much interest to a
service provider. Business value lies in the separation ofthe `what'
from the `how.' 

The question arises, are there any interesting or useful specification
mechanisms in between pure connectivity (WSDL) and full implementation
(BEPL,WSCI)? The answer is very positive, since the last 30 years of
research on the algebraic specification of behavior has yielded simple,
but powerful, methods for specifying the behavior of mobile and
distributed systems. In turn, this research has led to a program for the
development of type systems for the specification and automatic
verification of crucial properties of such systems. These properties
include both lock-freedom and leak-freedom. Of necessity, the
expressive powerof these types systems is below Turing-complete.

\section{Services, Ports, and Port Types}

The Web service notion provides an abstraction for autonomous
computational entities. Much as it may surprise some, the fundamental
abstraction is one based on message-passing. Even WSDL makes no
assumptions about synchrony, or RPC-like transports. Rather, it
proposes an ontology of services built on ports and messages. Messages
arrive at ports. Depending on their point of origin, they are either
received by the service or by the service's environment. That is,
programs and processes requiring access to a Web service access it by
sending messages to ports. Likewise, a Web service accesses the
programs and processes that make up its environment by sending
messages to ports.  

\subsection{Autonomy} 

This abstraction is appealing from two different vantage
points. First, the expanding commercialization of the Internet has
made it apparent that autonomy is a reality. Applications must be able
to interact across trust boundaries, for example, from shipper to
supplier. No single program is in charge.  The message-passing nature
of the Web services abstraction fits well with the reality of
autonomy.  Message-passing does not require more centralized resources
than are currently provided by the Internet
infrastructure. Message-passing, unlike shared memory, or
generalizations thereof, does not require a single, centralized store
to govern synchronization and consistency between concurrently
communicating applications.

\subsection{Programming Model Neutrality}

Secondly, the Web services abstraction normalizes different
programming models. Object-based models can be made to look like Web
services. Traditional EAI messaging applications can be made to look
like Web services. Databases can be made to look like Web
services. The different binding models of WSDL, as it was originally
conceived, support this point of view.  

This kind of programming model neutrality is essential. The increase
of connectivity amongst programs and processes brought on by the
expansion of the Internet has also entailed an increase in the
heterogeneity of any single process' environment. That is, a more
connected program is a more `worldly' program, more likely to be in
touch with programs and processes written in different styles, using
different operating assumptions.  

\subsection{Scale Invariance}

Additionally, this abstraction is neutral in regard to the scale of
application. In particular, we expect a business process with a
temporal extension of several months probably should be
message-driven. Similarly, OS-level components are often
message-driven. And, in a world where radio communication may
ultimately be cheaper than cable, it is reasonable to think of such
components as potentially separated by a network; and, an OS as a
collaborating set of services. So, the notion of Web service provides
a very simple ontology out of which a very rich and complicated set of
computational phenomena may be marshaled.  

\subsection{Port Types} 

Of course, this ontology is a little too simple. In practice we know
that not every kind of service is willing to handle any kind of
message sent to any port. Rather, a service expects that at particular
ports certain kinds of messages appear. That is, we expect to be able
to discipline the use of ports and need to be able to describe that
discipline so that programs wishing to access a service via such a
port may do so according to the discipline. 

These considerations give rise to the notion of a port type. In the
original WSDL specification the notion of port type ultimately
described a kind of sort or interface. That is, when all was said and
done, it described a collection of message types (XML schema, really)
that were expected to show up at the port, from one direction or the
other.

This sort of typing is useful and can help to catch many kinds of usage
errors at service binding time,as opposed to runtime. This alleviates
the programmer from writing tedious and repetitive code and eliminates
the introduction of many errors committed by getting this code
wrong. It also speeds up the runtime of service provision because these
checks may be done at service binding time.

\subsection{Types as Contracts}

In providing a way to describe these kinds of usage disciplines, the
WSDL notion of port type describes a kind of contract between service
requester and service provider. Of course, this is a very impoverished
notion of contract. It says really very little about quality of
service.It says very little about time. And, it says very little about
order. But, all of these things and many others would have to appear in
a business-level contract governing an actual service provision.  

Of course, the reason such information appears in such a contract is
that in a world of autonomous agents acting from either side of a trust
boundary, it is necessary to make implicit assumptions about service
provisioning explicit. A contract that merely says the service provider
will provide certain data upon receipt of a digitally signed request is
certainly met if the provider offers the data a quarter of a century
later.  

Similarly, we see that order, as well as time, is a critical aspect of
service level contracts between components, and one that applies at
varying temporal scales. For example, device drivers critically depend
on ordering and timing information. An OS can be brought to sudden and
abrupt halt if it and one of its device drivers differ in its
expectation on the order and timing of messages.  

Moreover, even though the WSDL port type does allow the contract to
specify the provider will provide such and such data upon receipt of a
digitally signed request it does not allow any other ordering
constraint to be specified. For example, if the protocol was initiated
by the service (`you have been selected from a pool of applicants to
make the next level request') soliciting the signed request and the
granting the data, WSDL simply remains silent. This kind of
protocollevel information must be specified elsewhere.  

\subsection{Scope: What Should A Contract Cover?} 

Since this contractual information is practical and necessary to
successfully implement and execute service provisioning in the real
world, ultimately, we claim, in a real contract language, there will
be a place for all of it: quality of service, duration, and
order. Here, however, we focus on logical constraints like order and
channel format because they have such profound impact on overall
service function and of the three kinds of information identified is
the most scale-invariant.  

There are several reasons for wishing to add these kinds of logical
constraints to port types. One arises from the practical need to make
more and more of the protocol semantics part of the mechanistically
enforced discipline on the use of the port. The extent to which it is
not is the extent to which runtime code is burdened with checks for
protocol violation.  

Another reason is that correct service functioning is really a matter
of disciplined communication. If the service requires an
initialization message before it is called upon to do work then
usually all bets are off if it processes work requests before being
initialized. Similarly, if a service is expecting to be notified that
an applicant wishes processing while the applicant is expecting to be
notified it may apply for processing, these two agents will make no
progress. These are instances of safety and liveness properties that
may be addressed if protocol-level information is added to the service
description.  

Finally, the more such contractual information can be made explicit
and mechanized the more service level agreements can be
automated. This has impact on the discovery process. If it can be
determined before the service is actually provisioned that there will
be a problem because the protocol expectations do not match between
service requester and service provider, then the service requester can
reject this service provider at discovery time. 

\subsection{Saying Too Much}

There is a central question at the heart of the Web services contracts
discussion: how much protocol information should be exposed at the
level of the public contract? Too much protocol information may lead
to feasibility and tractability issues. Too little and we will have
complicated Web service description for no recognizable gain.  

The critique levied against proposals such as BEPL4WS and WSCI may be
illustrated in terms of a game. Suppose the contract definition
language is a computationally complete programming language.  Player
(service requester) and opponent (service provider) want to engage in
business and want to do so in a timely fashion. Suppose that opponent
challenges player, indicating that if player wants to engage in
business it should download the following protocol definition,
implement, and execute it.  

A good sport, player does download the definition.  But, not entirely
gullible, player looks over the protocol and because the protocol
definition may be a program of arbitrary complexity, eventually times
out not completely satisfied that there are not clauses in the
definition that do not leave player in a disadvantageous
position. Still wishing to engage in business, though, player counters
with a protocol of its own devising, suggesting that because it cannot
ratify opponents proposal, opponent should adopt player's.  

Opponent recapitulates player's steps, and, after a certain amount of
time, is also unable to ascertain whether or not player's counter is
going to leave opponent in an awkward position. What is the
mechanistic procedure by which this game converges in a finitetime to a
protocol that both player and opponent can ratify? There is none. This
approach to the problem is not feasible. Contract languages that are
computationally complete -- and BEPL4WS and WSCI are just two of several
currently being considered for standardization -- will not work.

Instead, a different approach is required. Our contention is that Web
services are naturally implemented as processes as defined in the
process algebra literature, but that the properties one would like
to observe at the contractual level include, at least, those that may be
guaranteed by typing such processes. We observe that there are at least
four major classes of problems amenable to remedy by static checks over
descriptions of much simpler complexity than a complete programming
language. In particular, many aspects of safety, liveness, security
and resource management may be addressed by so-called behavioral types
[6, 7, 9].

\section{Services as Processes}

A comprehensive introduction to the theory of mobile processes is
beyond the scope of this article.  However, one of the great
advantages of basing an approach on this theory is that curious
readers can find a rich collection of literature on the subject; we
encourage investigating it. In particular, Milner's `The Polyadic
�-Calculus: A Tutorial' [8] is the best introduction to the subject
the authors have read.  Suffice it to say it is not an accident the
simple ontology adopted by Web services is the same base ontology
adopted for the algebraic treatment of mobile processes. This
correspondence between the Web service and process algebraic views of
the world is the result of an active and vigorous design process.
People involved in the development of one are involved in the
development of the other. For example, the first author of this
article is also co-author of the WSDL specification.  

To make things a little more concrete, the {\pic}-calculus (and its
relatives) are built around the abstraction of the port, often called
a name in that setting.  Processes are built in terms of
synchronization constraints over I/O requests (messages), at a
collection of ports (also called channels in the sequel). For example,
a mild variant of the {\pic}-calculus presented in Milner [8], offers the
classes of processes described by the EBNF grammar shown in the table
here.  

\section{Example: Initialization, Work, Finalization}

The following example provides a simplified implementation of a
service that expects to be initialized, then is willing to accept
repeated work requests or a finalize request. The initialization
request, which is sent on the `well-known' port web, is expected to
provide, in order, the initialization data, init, as well as two
ports, work and fin, where work and finalization requests are
sent. The port, x, merely represents data that might be sent in the
work and finalization requests. Note that on receiving a request on
the fin port the agent has no continuation, and therefore ceases to
interact: 

web(init,work,fin).(rec(k).(work(x).k + fin(x)) 

Rendering this in XML the reader would note a striking similarity to
BEPL4WS. Again, the similarity is not accidental. BEPL4WS drew much of
its inspiration from the version of XLANG that was used as the
internal processing language of the BizTalk engine. That language, the
principal designer of which is one of the authors, was explicitly
based on an asynchronous {\pic}-calculus.

Of course, this example is not chosen at random, either. Rather, it
abstracts a common pattern found in everyday programming: opening a
file or socket, reading and writing to it, repeatedly, and closing it;
opening a database connection, querying and updating over it,
repeatedly, and then closing it.  

A client of this service that makes two work requests before
finalizing would appear as in Figure 1a. Completing the example, the
system in which client talks to server would be written as shown in
Figure 1b. Note the use of the parallel composition to bring together
client and server. This technique is one of the hallmarks of process
algebras. It allows components to be defined independently and
executed autonomously and yet come together interactively; readers are
encouraged to consider how this example might be rendered in XML.


\section{Behavioral Types as Contracts}

\subsection{Port Types Versus Process Types}

One question to ask when considering behavioral types is simply `What
is typed?' -- there are at least two possible answers. In one case, only
the (collection of) port(s) is typed. Here, the notion of type may be
considered an extension of Milner's original notion of sort [8].  This
approach is conceptually simple and appealing.  The notion of sort,
for example, proscribes the shape of data that may flow over a port,
much like a data type might proscribe the shape of a
parameter. The usage types of Kobayashi [7] may be seen as
augmentations of the notion of sort to describe ordering constraints to
be respected by operations at the port.

But, it is not clear, for example, how a system guaranteeing
lock-freedom will address terms of the form xy.P0 + zw.P1. Because
interaction at the port x precludes interaction at z, and vice versa,
the choice forms an invisible information link between the two ports;
there is no place in a typing system based solely on port types to
record this link.  

In the other case, the entire process is typed.Here, the notion of
type is a homomorphic image ofthe process. In many systems, such as
Kobayashi and Ishiguro's generic type system [6], the type is
a process. In such systems there will invariably be a type compositor
that corresponds to the term compositor for parallel composition. This
compositor plays a key role in the semantics of the type
systems.

\subsection{Subject Reduction}

Behavioral types systems may also be classified by how they treat the
type compositor corresponding to the term compositor for parallel
composition (P0 P1 ).  Effectively, there are two possible
interpretations -- lazy and eager. In the lazy treatment, as in
Kobayashi's systems [6, 7],the type compositor mirrors precisely the
semanticsof the term compositor. No calculation of the `reductions' of
the types is effected in the resulting type. This design decision has
two consequences.  

First, the system must introduce a set of reduction rules on types
that parallels the reduction ruleson processes. Secondly, the subject
reduction theorem -- the theorem indicating how the evolution of processes
is related to their types -- is a kind of simulation. That is, the
subject reduction theorem is ofthe form 

\begin{equation*}

	P : T \and P \red P' \Rightarrow \exists T'. T \Rightarrow T' \and P' : T'

\end{equation*}

In other words, when the process $P$ is typed $T$ and the process can make
a move to $P'$, then there is a type $T'$ such that the type $T$ can move to
it and $P'$ types $T'$.  

In the eager treatment, the type compositor eagerly calculates what is
left of the types after `communication' (between the types)
occurs. Thus, in Yoshida, et al [9] the compositor $T_0 \odot T_1$ causes
the types of the ports involved in communication to reflect the
communication that would occur between the processes they type. This
has two corresponding consequences.  

First, the system must provide the definition of this compositor that
is, in some sense, outside the process algebra compositors. Secondly,
the subject reduction theorem is of the form 

\begin{equation*}

	P : T \and P \red P' \Rightarrow P' : T

\end{equation*}

In other words, once a process $P$ has a type $T$, then all evolutions
thereof have the same type.  

\subsection{Example: Generic Types}

Kobayashi and Igarashi develop a very general notion of behavioral
types they call process types [6]. The type system is for yet another
variant of the calculus than the one presented previously; the
analogous server implementation to the example can be expressed in
their calculus in the expression shown in Figure 2. His calculus is
minimal and offers a replication operator (!P) instead of a recursive
definition. Thus, the encoding is obliged to activate the recursive
invocations along a private channel, $s$; their type judgments are of
the following form:$ \Gamma \vdash P$. The type environment, $\Gamma$,
types the entire process. The judgment may be read as `$\Gamma$
abstracts the process, $P$.' The example types are shown in Figure
3. Ignoring the funny shape of the inputs, the expression to the left
of the turnstile in Figure 3 (that is, the type), can indeed be seen
as an abstraction of the process -- note that the private channel
activations disappear in the type.

In this setting, well-typedness is parameterized by a user-supplied
predicate ok. Depending on the shape of the predicate, well-typedness
guarantees a range of properties from lock-freedom to resource
management in the form of static garbage-channel collection. It is
enlightening to investigate the typing of a client of this service
using different predicates.  

The example system illustrates there are two different kinds of
relationships that are governed by typing. One is evident in the shape
of the judgment.  It answers the question: does the service provider
service actually provide the service advertised by the type? We call
this relationship conformance.  

The other relationship is more subtle. It answers the question of
whether a pair of types match up so that these patterns of behavior
allow successful interoperation. Again, this is realized differently
depending on way the type system treats parallel composition. In an
eager treatment, undesirable interoperation will occur early in the
composition oftypes. In the lazy treatment it will show up as a
property of the (evolution of the) parallel composition ofthe
types. We call this relationship compatibility.  

In this regard, the system also illustrates the flaw in a common
misconception regarding compatibility. A cursory investigation might
lead one to think thesetype systems require `matching' is one-for-one
dual behavior on either side of a session. Kobayashi's system, however,
has no such duality. Parallel composition of services may `leave
something over' so that the aggregate service may have behavior it
exposes to a third party, as we might see in a practical
system involving a client, a supplier, and a shipper.  

The usual notion of program correctness -- a program meets a
specification -- corresponds to the `local' notion of
conformance. End-to-end correctness is accommodated by successful
answers to questions posed by both relationships. Specifically,
client must conform to its type. Server must conform to its type. The
type of client must be compatible with the type of server. However, it
is important to realize that in systems built out of the composition of
other systems checking conformance entails checking compatibility.

Interestingly, service discovery may be favorably impacted by the
notion of compatibility. As many have observed, a WSDL-based discovery
is ultimately based on service names. This means knowledge of how the
service name relates to what it does must be magically applied to the
search -- this usually means a human is involved. When compatibility can
be mechanized some of the chores of searching for the right service can
be off-loaded to machines. Given that schema for broad classes of
service requests (purchase orders, advanced shipping notices) are
moving toward standardization, what can be mechanized is that services
using these standard schemas exchange documents typed by them without
locking up orleaking secrets or resources.  

\section{Conclusion}

Magazine space constraints preclude an exhaustive survey of existing
behavioral type systems here. As mentioned earlier, systems of more or
less comparable complexity exist for guaranteeing properties ranging
from lock-freedom to leak-freedom. The simplicity of these systems
makes rendering them in XML schema routine. Their simplicity,
however, belies their value. It is of considerable business value to be
certain that when a client binds to a service it has discovered on the
Web it can be assured it will not lock up, nor will it leak resources
or secrets.These sorts of guarantees are more important to thebottom
line than a description of the implementation of the client or the
service.  

In fact, descriptions based on behavioral types canform the basis of a
new kind of discovery mechanism, specifically one based on partial
behavioral descriptions. This mechanism crucially depends on the
computation of compatibility of types being a terminating
computation. Turing-complete service description languages will face
difficulties filling this niche in the emerging Web services ecosystem.

Of course, a caveat remains. Whatever is chosen as the basis of the
contractual mechanism, it is still a matter of trust, from client's
point of view, that a given service actually conforms to the contract
it advertises. This suggests even higher level services, for
example,so-called reputation services analogous to business rating
services like Dun and Bradstreet, will be needed toguarantee
end-to-end quality of service.
